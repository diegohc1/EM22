% Options for packages loaded elsewhere
\PassOptionsToPackage{unicode}{hyperref}
\PassOptionsToPackage{hyphens}{url}
%
\documentclass[
]{article}
\usepackage{amsmath,amssymb}
\usepackage{lmodern}
\usepackage{iftex}
\ifPDFTeX
  \usepackage[T1]{fontenc}
  \usepackage[utf8]{inputenc}
  \usepackage{textcomp} % provide euro and other symbols
\else % if luatex or xetex
  \usepackage{unicode-math}
  \defaultfontfeatures{Scale=MatchLowercase}
  \defaultfontfeatures[\rmfamily]{Ligatures=TeX,Scale=1}
\fi
% Use upquote if available, for straight quotes in verbatim environments
\IfFileExists{upquote.sty}{\usepackage{upquote}}{}
\IfFileExists{microtype.sty}{% use microtype if available
  \usepackage[]{microtype}
  \UseMicrotypeSet[protrusion]{basicmath} % disable protrusion for tt fonts
}{}
\makeatletter
\@ifundefined{KOMAClassName}{% if non-KOMA class
  \IfFileExists{parskip.sty}{%
    \usepackage{parskip}
  }{% else
    \setlength{\parindent}{0pt}
    \setlength{\parskip}{6pt plus 2pt minus 1pt}}
}{% if KOMA class
  \KOMAoptions{parskip=half}}
\makeatother
\usepackage{xcolor}
\usepackage[margin=1in]{geometry}
\usepackage{graphicx}
\makeatletter
\def\maxwidth{\ifdim\Gin@nat@width>\linewidth\linewidth\else\Gin@nat@width\fi}
\def\maxheight{\ifdim\Gin@nat@height>\textheight\textheight\else\Gin@nat@height\fi}
\makeatother
% Scale images if necessary, so that they will not overflow the page
% margins by default, and it is still possible to overwrite the defaults
% using explicit options in \includegraphics[width, height, ...]{}
\setkeys{Gin}{width=\maxwidth,height=\maxheight,keepaspectratio}
% Set default figure placement to htbp
\makeatletter
\def\fps@figure{htbp}
\makeatother
\setlength{\emergencystretch}{3em} % prevent overfull lines
\providecommand{\tightlist}{%
  \setlength{\itemsep}{0pt}\setlength{\parskip}{0pt}}
\setcounter{secnumdepth}{-\maxdimen} % remove section numbering
\ifLuaTeX
  \usepackage{selnolig}  % disable illegal ligatures
\fi
\IfFileExists{bookmark.sty}{\usepackage{bookmark}}{\usepackage{hyperref}}
\IfFileExists{xurl.sty}{\usepackage{xurl}}{} % add URL line breaks if available
\urlstyle{same} % disable monospaced font for URLs
\hypersetup{
  pdftitle={EM 2022 - Reporte descriptivo de los cuestionarios de factores asociados},
  pdfauthor={Oficina de Medición de la Calidad de los Aprendizajes (UMC)},
  hidelinks,
  pdfcreator={LaTeX via pandoc}}

\title{EM 2022 - Reporte descriptivo de los cuestionarios de factores
asociados}
\author{Oficina de Medición de la Calidad de los Aprendizajes (UMC)}
\date{15/03/2023}

\begin{document}
\maketitle

{
\setcounter{tocdepth}{3}
\tableofcontents
}
\pagebreak

\hypertarget{introducciuxf3n}{%
\section{Introducción}\label{introducciuxf3n}}

El presente documento tiene por objetivo reportar los resultados
descriptivos de los cuestionarios aplicados a directores, docentes,
estudiantes y padres de familia en el marco de la Evaluación Muestral
2019 (EM 2019)

Los cuestionarios de la EM 2019 contaron con un conjunto de preguntas
que consideraron tanto características como percepciones de los actores
que participaron en el estudio. Los cuestionarios incluyeron preguntas
que conforman indicadores simples (p.~ej. sexo), una serie de preguntas
que representan constructos latentes (p.~ej. agotamiento emocional) y
preguntas abiertas (p.~ej. ¿Qué opinión tiene sobre ``Aprendo en
casa''?). En este documento, se reporta las respuestas de los
indicadores simples e indicadores que conforman un constructo.

\pagebreak

\hypertarget{cuestionario-a-la-instituciuxf3n-educactiva-2s}{%
\section{Cuestionario a la Institución Educactiva
2S}\label{cuestionario-a-la-instituciuxf3n-educactiva-2s}}

\hypertarget{resultados-descriptivos}{%
\subsubsection{Resultados descriptivos}\label{resultados-descriptivos}}

\hypertarget{p01.-sexo}{%
\subsubsection{p01. Sexo}\label{p01.-sexo}}

\includegraphics[width=0.4\textwidth,height=\textheight]{D:/EM22/02-reportes-scripts/01-descriptivos/figuras-para-reportes/00-nacional/g_EM2022_2Sdirector_EBR_NA.pdf}

\hypertarget{p02.-quuxe9-cargo-ocupa-usted-en-esta-escuela}{%
\subsubsection{p02. ¿Qué cargo ocupa usted en esta
escuela?}\label{p02.-quuxe9-cargo-ocupa-usted-en-esta-escuela}}

\includegraphics[width=0.65\textwidth,height=\textheight]{D:/EM22/02-reportes-scripts/01-descriptivos/figuras-para-reportes/00-nacional/g_EM2022_2Sdirector_EBR_p01.pdf}

\hypertarget{p03.-cuxf3mo-obtuvo-usted-su-tuxedtulo-pedaguxf3gico}{%
\subsubsection{p03. ¿Cómo obtuvo usted su título
pedagógico?}\label{p03.-cuxf3mo-obtuvo-usted-su-tuxedtulo-pedaguxf3gico}}

\includegraphics[width=0.4\textwidth,height=\textheight]{D:/EM22/02-reportes-scripts/01-descriptivos/figuras-para-reportes/00-nacional/g_EM2022_2Sdirector_EBR_p02.pdf}

\hypertarget{p04.-cuuxe1l-es-el-muxe1ximo-nivel-educativo-que-usted-ha-alcanzado}{%
\subsubsection{p04. ¿Cuál es el máximo nivel educativo que usted ha
alcanzado?}\label{p04.-cuuxe1l-es-el-muxe1ximo-nivel-educativo-que-usted-ha-alcanzado}}

\includegraphics[width=0.65\textwidth,height=\textheight]{D:/EM22/02-reportes-scripts/01-descriptivos/figuras-para-reportes/00-nacional/g_EM2022_2Sdirector_EBR_p03.pdf}

\hypertarget{p07.-en-toda-su-carrera-se-ha-capacitado-sobre-los-siguientes-temas-durante-su-formaciuxf3n-inicial-o-durante-su-tiempo-en-servicio}{%
\subsubsection{p07. En toda su carrera, ¿se ha capacitado sobre los
siguientes temas durante su formación inicial o durante su tiempo en
servicio?}\label{p07.-en-toda-su-carrera-se-ha-capacitado-sobre-los-siguientes-temas-durante-su-formaciuxf3n-inicial-o-durante-su-tiempo-en-servicio}}

\includegraphics{D:/EM22/02-reportes-scripts/01-descriptivos/figuras-para-reportes/00-nacional/g_EM2022_2Sdirector_EBR_p04.pdf}

\hypertarget{p08.-en-su-escuela-en-quuxe9-modalidad-se-brindaron-las-clases-a-los-estudiantes-en-los-uxfaltimos-auxf1os}{%
\subsubsection{p08. En su escuela, ¿en qué modalidad se brindaron las
clases a los estudiantes en los últimos
años?}\label{p08.-en-su-escuela-en-quuxe9-modalidad-se-brindaron-las-clases-a-los-estudiantes-en-los-uxfaltimos-auxf1os}}

\includegraphics{D:/EM22/02-reportes-scripts/01-descriptivos/figuras-para-reportes/00-nacional/g_EM2022_2Sdirector_EBR_p07.pdf}

\hypertarget{p09.-en-su-escuela-se-realizaron-las-siguientes-actividades-durante-el-peruxedodo-de-recuperaciuxf3n-durante-los-meses-de-enero-y-febrero-2022}{%
\subsubsection{p09. En su escuela, ¿se realizaron las siguientes
actividades durante el período de recuperación durante los meses de
enero y febrero
2022?}\label{p09.-en-su-escuela-se-realizaron-las-siguientes-actividades-durante-el-peruxedodo-de-recuperaciuxf3n-durante-los-meses-de-enero-y-febrero-2022}}

\includegraphics{D:/EM22/02-reportes-scripts/01-descriptivos/figuras-para-reportes/00-nacional/g_EM2022_2Sdirector_EBR_p08.pdf}

\hypertarget{p10.-al-inicio-del-auxf1o-escolar-se-aplicuxf3-la-evaluaciuxf3n-diagnuxf3stica-de-entrada-a-los-estudiantes-de-su-escuela}{%
\subsubsection{p10. Al inicio del año escolar, ¿se aplicó la evaluación
diagnóstica de entrada a los estudiantes de su
escuela?}\label{p10.-al-inicio-del-auxf1o-escolar-se-aplicuxf3-la-evaluaciuxf3n-diagnuxf3stica-de-entrada-a-los-estudiantes-de-su-escuela}}

\includegraphics{D:/EM22/02-reportes-scripts/01-descriptivos/figuras-para-reportes/00-nacional/g_EM2022_2Sdirector_EBR_p09.pdf}

\hypertarget{p11.-seguxfan-los-resultados-de-la-evaluaciuxf3n-diagnuxf3stica-de-entrada-los-docentes-han-planificado-alguna-de-las-siguientes-actividades-para-apoyar-a-los-que-no-han-alcanzado-los-aprendizajes-esperados}{%
\subsubsection{p11. Según los resultados de la evaluación diagnóstica de
entrada, ¿los docentes han planificado alguna de las siguientes
actividades para apoyar a los que no han alcanzado los aprendizajes
esperados?}\label{p11.-seguxfan-los-resultados-de-la-evaluaciuxf3n-diagnuxf3stica-de-entrada-los-docentes-han-planificado-alguna-de-las-siguientes-actividades-para-apoyar-a-los-que-no-han-alcanzado-los-aprendizajes-esperados}}

\includegraphics{D:/EM22/02-reportes-scripts/01-descriptivos/figuras-para-reportes/00-nacional/g_EM2022_2Sdirector_EBR_p10.pdf}

\hypertarget{p12.-durante-el-presente-auxf1o-escolar-despuuxe9s-de-implementarse-las-clases-presenciales-al-100-una-o-algunas-secciones-de-2.-grado-de-secundaria-tuvieron-que-enseuxf1ar-a-distancia-por-presencia-de-covid-19}{%
\subsubsection{p12. Durante el presente año escolar, después de
implementarse las clases presenciales al 100\%, ¿una o algunas secciones
de 2°. grado de secundaria tuvieron que enseñar a distancia por
presencia de
Covid-19?}\label{p12.-durante-el-presente-auxf1o-escolar-despuuxe9s-de-implementarse-las-clases-presenciales-al-100-una-o-algunas-secciones-de-2.-grado-de-secundaria-tuvieron-que-enseuxf1ar-a-distancia-por-presencia-de-covid-19}}

\includegraphics[width=0.65\textwidth,height=\textheight]{D:/EM22/02-reportes-scripts/01-descriptivos/figuras-para-reportes/00-nacional/g_EM2022_2Sdirector_EBR_p11.pdf}

\hypertarget{p13.-en-el-presente-auxf1o-escolar-con-quuxe9-frecuencia-los-docentes-de-secundaria-han-realizado-las-siguientes-actividades}{%
\subsubsection{p13. En el presente año escolar, ¿con qué frecuencia los
docentes de secundaria han realizado las siguientes
actividades?}\label{p13.-en-el-presente-auxf1o-escolar-con-quuxe9-frecuencia-los-docentes-de-secundaria-han-realizado-las-siguientes-actividades}}

\includegraphics{D:/EM22/02-reportes-scripts/01-descriptivos/figuras-para-reportes/00-nacional/g_EM2022_2Sdirector_EBR_p12.pdf}

\hypertarget{p14.-seguxfan-su-opiniuxf3n-respecto-a-los-docentes-de-la-instituciuxf3n-educativa-que-usted-dirige-quuxe9-tan-de-acuerdo-estuxe1-con-las-siguientes-afirmaciones}{%
\subsubsection{p14. Según su opinión respecto a los docentes de la
institución educativa que usted dirige, ¿qué tan de acuerdo está con las
siguientes
afirmaciones?}\label{p14.-seguxfan-su-opiniuxf3n-respecto-a-los-docentes-de-la-instituciuxf3n-educativa-que-usted-dirige-quuxe9-tan-de-acuerdo-estuxe1-con-las-siguientes-afirmaciones}}

\includegraphics{D:/EM22/02-reportes-scripts/01-descriptivos/figuras-para-reportes/00-nacional/g_EM2022_2Sdirector_EBR_p13.pdf}

\hypertarget{p15.-en-el-presente-auxf1o-escolar-con-quuxe9-frecuencia-usted-se-ha-comunicado-con-los-siguientes-actores-con-el-propuxf3sito-de-intercambiar-ideas-u-opiniones-sobre-cuxf3mo-atender-alguxfan-problema-en-la-instituciuxf3n-educativa-que-dirige}{%
\subsubsection{p15. En el presente año escolar, ¿con qué frecuencia
usted se ha comunicado con los siguientes actores con el propósito de
intercambiar ideas u opiniones sobre cómo atender algún problema en la
institución educativa que
dirige?}\label{p15.-en-el-presente-auxf1o-escolar-con-quuxe9-frecuencia-usted-se-ha-comunicado-con-los-siguientes-actores-con-el-propuxf3sito-de-intercambiar-ideas-u-opiniones-sobre-cuxf3mo-atender-alguxfan-problema-en-la-instituciuxf3n-educativa-que-dirige}}

\includegraphics{D:/EM22/02-reportes-scripts/01-descriptivos/figuras-para-reportes/00-nacional/g_EM2022_2Sdirector_EBR_p14.pdf}

\hypertarget{p16.-seguxfan-el-contexto-en-el-que-se-encuentra-el-docente-pueden-haber-diferentes-formas-de-enseuxf1ar.-en-su-opiniuxf3n-quuxe9-tan-de-acuerdo-estuxe1-con-los-siguientes-enunciados}{%
\subsubsection{p16. Según el contexto en el que se encuentra el docente
pueden haber diferentes formas de enseñar. En su opinión, ¿qué tan de
acuerdo está con los siguientes
enunciados?}\label{p16.-seguxfan-el-contexto-en-el-que-se-encuentra-el-docente-pueden-haber-diferentes-formas-de-enseuxf1ar.-en-su-opiniuxf3n-quuxe9-tan-de-acuerdo-estuxe1-con-los-siguientes-enunciados}}

\includegraphics{D:/EM22/02-reportes-scripts/01-descriptivos/figuras-para-reportes/00-nacional/g_EM2022_2Sdirector_EBR_p15.pdf}

\hypertarget{p17.-en-las-uxfaltimas-dos-semanas-con-quuxe9-frecuencia-se-ha-sentido-de-la-siguiente-manera}{%
\subsubsection{p17. En las últimas dos semanas, ¿con qué frecuencia se
ha sentido de la siguiente
manera?}\label{p17.-en-las-uxfaltimas-dos-semanas-con-quuxe9-frecuencia-se-ha-sentido-de-la-siguiente-manera}}

\includegraphics{D:/EM22/02-reportes-scripts/01-descriptivos/figuras-para-reportes/00-nacional/g_EM2022_2Sdirector_EBR_p16.pdf}

\hypertarget{section}{%
\subsubsection{.}\label{section}}

\end{document}
